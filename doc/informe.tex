% ******************************************************** %
%              TEMPLATE DE INFORME                         %
% ******************************************************** %
% ******************************************************** %
%                                                          %
% ALGUNOS PAQUETES REQUERIDOS (EN UBUNTU):                 %
% ========================================
%                                                          %
% texlive-latex-base                                       %
% texlive-latex-recommended                                %
% texlive-fonts-recommended                                %
% texlive-latex-extra?                                     %
% texlive-lang-spanish (en ubuntu 13.10)                   %
% ******************************************************** %


\documentclass[a4paper]{article}
\usepackage[spanish]{babel}
\usepackage[utf8]{inputenc}
\usepackage{charter}   % tipografia
\usepackage{graphicx}
\usepackage[table,xcdraw]{xcolor}
\usepackage{hyperref}
%\usepackage{makeidx}
\usepackage{paralist} %itemize inline

\usepackage{float}
\usepackage{amsmath, amsthm, amssymb}
\usepackage{amsfonts}
%\usepackage{sectsty}
%\usepackage{charter}
%\usepackage{wrapfig}
\usepackage{listingsutf8}

% \setcounter{secnumdepth}{2}
\usepackage{underscore}
\usepackage{caratula}
\usepackage{url}
%\usepackage[superscript,biblabel]{cite}
%\usepackage{dibujitos}

% ********************************************************* %
% ~~~~~~~~              Code snippets             ~~~~~~~~~ %
% ********************************************************* %

\usepackage{color} % para snipets de codigo coloreados
\usepackage{fancybox}  % para el sbox de los snipets de codigo

\definecolor{litegrey}{gray}{0.94}

\newenvironment{codesnippet}{%
	\begin{Sbox}\begin{minipage}{\textwidth}\sffamily\small}%
	{\end{minipage}\end{Sbox}%
		\begin{center}%
		\vspace{-0.4cm}\colorbox{litegrey}{\TheSbox}\end{center}\vspace{0.3cm}}

\definecolor{mygreen}{rgb}{0,0.6,0}
\definecolor{mygray}{rgb}{0.5,0.5,0.5}
\definecolor{mymauve}{rgb}{0.58,0,0.82}

\lstset{ %
  backgroundcolor=\color{litegrey},
  basicstyle=\footnotesize,
  breakatwhitespace=true,
  breaklines=true,
  captionpos=b,                    % sets the caption-position to bottom
  mathescape=true,
  keepspaces=true,
  language=Python,
  showspaces=false,
  tabsize=2,                       % sets default tabsize to 2 spaces
  inputencoding=utf8/latin1
}

\newcommand{\cuidado}{{\large $\Delta$!!!} \hspace*{1em}}

% ********************************************************* %
% ~~~~~~~~         Formato de las páginas         ~~~~~~~~~ %
% ********************************************************* %

\usepackage{fancyhdr}
\pagestyle{fancy}

\renewcommand{\sectionmark}[1]{\markright{\thesection\ - #1}}

\fancyhf{}

\fancyhead[LO]{Sección \rightmark} % \thesection\
\fancyfoot[LO]{\small{}} % alumnos
\fancyfoot[RO]{\thepage}
\renewcommand{\headrulewidth}{0.5pt}
\renewcommand{\footrulewidth}{0.5pt}
\setlength{\hoffset}{-0.8in}
\setlength{\textwidth}{16cm}
%\setlength{\hoffset}{-1.1cm}
%\setlength{\textwidth}{16cm}
\setlength{\headsep}{0.5cm}
\setlength{\textheight}{25cm}
\setlength{\voffset}{-0.7in}
\setlength{\headwidth}{\textwidth}
\setlength{\headheight}{13.1pt}

\renewcommand{\baselinestretch}{1.1}  % line spacing

% ******************************************************** %


\begin{document}


\thispagestyle{empty}
\materia{Problemas, Algoritmos y Programación}
\submateria{Primer Cuatrimestre de 2016}
\titulo{Trabajo Práctico 1}
\subtitulo{}
\integrante{}{}{}
\integrante{}{}{}
\integrante{}{}{}
\integrante{}{}{}

\maketitle
\newpage

\thispagestyle{empty}
\vfill

\thispagestyle{empty}
\vspace{2cm}
\tableofcontents
\newpage


\normalsize
\newpage
\section{Ejercicio 1}

En este ejercicio se pide la selección de un subconjunto de números de manera tal que su suma sea la más cercana a un número P dado.
Dicha suma debe ser menor o igual a P.

Podemos notar que es una variación del problema clásico conocido como \textbf{Subset Sum} y por lo tanto probaremos que se puede resolver con la misma complejidad.

El algoritmo consiste en dividir la entrada en dos subgrupos de tamaño $\lceil n/2 \rceil$ y $\lfloor n/2 \rfloor$.
Para el primer subgrupo calcular todas las posibles sumas para todos los subconjuntos posibles ($2^{n/2}$) y guardas las que sean menores a P en un conjunto que permita la consulta de la existencia de un elemento o bien que nos devuelva el primero más chico al buscado.

Para el segundo subgrupo, nuevamente calcular todas las posibles sumas para todos los subconjuntos posibles y en caso de ser menor estricto a P, buscar en la estructura anterior el primer número que sea menor o igual a (P - la suma acumulada). Ese número que nos devuelva sumado a la suma que nos dio el subconjunto calculado son una posible respuesta. 

Con el almacenamiento de una variable global que posea el máximo número alcanzado hasta el momento, ante cada consulta anterior me quedo siempre con el máximo hasta el momento y el nuevo valor encontrado. 

Basta con reportar esta variable global que es la respuesta al problema que estábamos buscando.

\subsection{Cómo cumplir con las complejidades pedidas}

Para lograr la complejidad de \bigo($n * 2 ^{n/2}$) basta con que la primer y la segunda parte del algoritmo tarden dicha complejidad.
La primera parte itera sobre $2 ^{n/2}$ subconjuntos y en cada uno de ellos calcula n/2 sumas. A su vez inserta en un conjunto (que lo implementaremos con un set de C++) el resultado de la suma. Esta última operación tarda \bigo(log M) donde M es la cantidad de elementos del conjunto. Como en nuestro conjunto tendremos almacenados hasta $2 ^{n/2}$ elementos su logaritmo es n/2 y entonces por cada iteración del ciclo se realizan en promedio n operaciones.
Con esto concluimos que la primer parte del algoritmo tiene complejidad  \bigo($n * 2 ^{n/2}$).

Para la segunda parte del algoritmo nuevamente tenemos $2 ^{n/2}$ iteraciones para las cuales haremos la misma cantidad de sumas que antes (n/2) y se deberá hacer una búsqueda sobre el conjunto. En caso de que la suma parcial ya sea mayor al target, ya no sirve continuar porque nos pasamos. Caso contrario, usamos la operación que nos provee el conjunto implementado sobre set llamada upper_bound que en tiempo logarítmico devuelve el primer elemento $>$ al target buscado. Como buscamos el primer elemento que sea menor o igual, retrocedemos en uno el iterador y llegamos al elemento que estábamos buscando (vale aclarar que como siempre está el número 0 en el conjunto, podemos asegurar que siempre habrá un elemento menor o igual). Estas dos operaciones tardan tiempo logarítmico sobre la cantidad de elementos del conjunto, y como bien habíamos dicho en el párrafo anterior la cantidad total era de $2 ^{n/2}$ elementos y entonces su logaritmo es n/2. Nuevamente tardamos \bigo($n * 2 ^{n/2}$).

El algoritmo termina con la ejecución de esos dos ciclos y la devolución de la respuesta en \bigo(1) por lo que la complejidad total se resume en \bigo($n * 2 ^{n/2}$).

Para testear el algoritmo se probaron todas instancias aleatorias para cada uno de los n posibles.

\section{Ejercicio 2}

\subsection{Enunciado}

El problema que se nos plantea es el siguiente. Nos pasan una lista de las direcciones de correo electrónico de todos los alumnos que cursarán PAP el cuatrimestre que viene, omitiendo la última parte que es común a todos (@dc.uba.ar). Además, para cada una de estas direcciones, recibimos un prefijo de esta (en realidad, el número de caracteres del prefijo, que es equivalente).
Nos piden calcular cuál es el mínimo valor de T que cumpla que para cada dirección
de correo electrónico que les pasemos junto con su prefijo, existen a lo sumo T direcciones que comparten ese prefijo.

La entrada es una cantidad A de direcciones, y luego direcciones seguidas de su prefijo, de la forma que ya especificamos.

\subsection{Reduccion del problema}

Tenemos entonces que para cada i en el rango de 1 a A, la $direccion_i$ y el $P_i$, el tamaño del prefijo.
Queremos ver como encontrar el número T pedido por el enunciado.
\par{Definimos entonces un $T_i$ para cada i, como la cantidad de direcciones que comparten el prefijo indicado por $P_i$. }
\par{Supongamos que T es la respuesta al problema, con T $<$ $T_i$ para algún i en el rango 1 a A. Entonces para ese i, tenemos $T_i$ direcciones que comparten el prefijo, y es mayor que T. Pero el problema nos pedia que T cumpla con que a lo sumo T direcciones comparten el prefijo. Absurdo.
Entonces sabemos que T $\geq$ $T_i$. El mínimo numero que cumple esta desigualdad es T = max\{$T_i$\}.} \par{Si probamos que este T cumple lo pedido, 
entonces es, por definición, el mínimo valor que cumple que para cada dirección
de correo electrónico que les pasemos junto con su prefijo, existen a lo sumo T direcciones que comparten ese prefijo.}
\par{Podemos ver que, como T $\geq$ $T_i$, la cantidad de direcciones que comparten cualquier prefijo de los indicados, es a lo sumo T.
Por lo cual este T efectivamente cumple lo pedido. 
Como comentario, lo que probamos es que un T esta en el conjunto al cual le tomamos mínimo, sí y solo sí T $\geq$ $T_i$.}

\subsection{Solución}
\par{Lo que nos queda entonces es, conseguir, para cada i en el rango 1 a A, el número $T_i$.
Es decir, queremos la cantidad de direcciones que comparten el prefijo indicado.}
\par{Para esto, recordaremos de Algoritmos y Estructuras de Datos II, la estructura \quotes{Trie}.
Recordemos que un Trie es un arbol que nos provee una representación eficiente de conjuntos o diccionarios cuyas entradas son Cadenas de texto. En un Trie, tendremos a cada arista del arbol que formamos se le asigna un caracter del diccionario. Luego, un nodo cualquiera del arbol esta en correspondencia directa con un prefijo del conjunto, ya que podemos subir a los padres hasta llegar a la raiz, para saber a que prefijo se corresponde ese nodo. Le asignamos al nodo raíz el prefijo vacío.}
\par{Además, las operaciones de inserción y consulta en un Trie, son proporcionales al largo del texto
insertado/consultado. En este caso, \bigo($\mid D_i \mid$). Por último, durante la inserción en un Trie, podemos llevar la cuenta de cuantas cadenas insertadas usan este nodo. Esto nos diría cuantas cadenas
tienen como prefijo al mismo prefijo representado por el nodo.}
\par{Por ende, para calcular $T_i$, debemos acceder a este valor que vamos a calcular en las inseraciones, que esta en el nodo correspondiente. Para acceder al nodo, lo consultamos como cuando consultamos un valor en un diccionario sobre Trie. Luego esta consulta va a ser \bigo($P_i$), donde
\bigo($P_i$) $\subseteq$ \bigo($\mid D_i \mid$), ya que $P_i$ $<$ $\mid D_i \mid$ . Por lo cual, decimos que esta consulta es \bigo($\mid D_i \mid$)}

\subsection{Implementación}
Vamos a dar una implementación de Trie no habitual. Definimos un struct Trie, en el cual
vamos a tener punteros a los hijos. Pero en lugar de manejar esos punteros de forma implícita, 
como es habitual, simplemente tenemos una estructura de diccionario de caracter a estructura Trie.
Esta estructura utiliza internamente memoria dinámica, así que nos almacenara implícitamente 
los punteros a estructuras de tipo Trie. Cuando hacemos un acceso, simplemente buscamos un valor
del diccionario. Map de la STL de C++ nos permite accesos en \bigo(log N), donde N es la cantidad de claves del diccionario. Como en este caso, el mayor N es el tamaño del diccionario, y este esta
acotado por ser cadenas de texto, entonces tenemos acceso en \bigo(log Tamaño dicc) = \bigo(1).
Por lo cual esta estructura se comporta igual que la clásica implementacion sobre arreglos.
Además tiene la ventaja de que al utilizar el operador \quotes{[]}(corchetes) para acceder a un elemento, si este no existe, entonces se crea una entrada con valor por defecto. Y en este caso, 
el valor por defecto es nodo del Trie vacío, por lo cual funciona correctamente.


\lstset{language=C++,
           basicstyle=\ttfamily,
           keywordstyle=\color{blue}\ttfamily,
           stringstyle=\color{red}\ttfamily,
           commentstyle=\color{green}\ttfamily,
          breaklines=true,
          stepnumber=1,
          tabsize=2}

\begin{lstlisting}
	
struct Trie {
	map<char, Trie> hijos;
	
	int contador;
};

\end{lstlisting}

\subsection{Pseudo-C++}

\begin{algorithmic}
	
	\Function{Agregar}{Trie nodo, string s, int i}
		\State nodo.contador++
		\If {i == s.size()}
			\State \Return
		\Else
			\State agregar(nodo.hijos[s[i]], s, i+1);
		\EndIf

	\EndFunction
	
	\vspace{2em}

	\Function{Contador}{Trie nodo, string s, int prefijo, int i}
		\If {i == prefijo}
			\State \Return nodo.contador
		\Else
			\State \Return contador(nodo.hijos[s[i]], s, prefijo, i+1)
		\EndIf
	\EndFunction

	\vspace{2em}

	\Function{Main}{}
		\State root $\gets$ Trie() \Comment Inicializo el Trie vacío

		\State A $\gets$ leer_input()

		\For{i = 0 to A}
			\State $D_i$ $\gets$ leer_input()
			\State Agregar(root, d, 0);

		\EndFor

		\For{i = 0 to A}
			\State $T_i$ $\gets$ contador(root, $D_i$, $P_i$, 0)
		\EndFor

		\State T $\gets$ max\{$T_i$\}

		\State Imprimir(T)

	\EndFunction





\end{algorithmic}

\subsection{Complejidad}

Como dijimos antes, todas las operaciones sobre el Trie son \bigo($\mid D_i \mid$). Hacemos una insercion y una consulta sobre cada $D_i$, con lo cual la complejidad nos queda
\bigo($\mid D_1 \mid$ + ... + $\mid D_A \mid$ ), que es la complejidad requerida por el enunciado. 
\section{Ejercicio 3}

El ejercicio consiste en encontrar la máxima suma de elementos contiguos de un array.
Esto es equivalente a encontrar un subarray que maximice la suma de sus elementos,
lo que se conoce como \textbf{Maximum Subarray Problem}.


Formalmente, dado un array $A = (a_0 \ldots a_{n-1})$ queremos encontrar el máximo de
$$
\sum_{s=i}^{j-1} a_s \;\;\;\; \text{con} \; 0 \leq i \leq j \leq n
$$


\subsection{Algoritmo}

La idea es recorrer los prefijos $A_k = (a_0 \ldots a_{k-1})$ en orden de $k$ creciente
y para cada uno calcular el máximo de la suma dentro del array, $res_k$,
y la suma máxima entre los subarrays que terminen en el elemento $k$ o el subarray vacío, $border_k$.

Para el caso $k=0$, el array vacío, ambos valores son 0.

Para el caso $k+1$, $res_{k+1}$ es el el máximo entre $res_k$ o el resultado
de sumarle $a_k$ a $border_k$, el máximo anterior usando hasta el último elemento.
$border_{k+1}$ será el máximo entre 0 (pues siempre podemos usar el array vacío) o $border_k + a_k$.

Como $A = A_n$, nuestro resultado final será $res_n$.

\subsection{Pseudocódigo}

\begin{algorithmic}

\Function{sumaMaxima}{$array$}
    \State $res \gets 0$
    \State $border \gets 0$
    \For{$x$ in $array$}
        \State $border \gets max(0, border+x)$
        \State $res \gets max(res, border)$
    \EndFor
    \State \Return{$res$}
\EndFunction

\end{algorithmic}

\subsection{Complejidad}

El recorre una vez el array calculando los máximos descriptos en \bigo(1), resultando en una complejidad total de \bigo($n$).


\section{Ejercicio 4}

En este ejercicio se pide que ante una serie de consultas a intervalos de un arreglo, se devuelva para cada una la suma de los dos elementos más grandes.

\subsection{Modelado}

Para poder responder las consultas de manera eficiente se decidió el uso de un segment tree con la operación ``máximo'' asociada.\newline

A diferencia de un segment tree clásico se decidió que no solo se guarde el valor máximo para el rango en cada uno de los nodos, sino también el índice a la posición del arreglo original que contiene dicho valor. Esto se debe a que se utilizará al momento de resolver la pregunta.

\subsection{Resolviendo las preguntas}

Las preguntas que nos piden son del tipo [a,b) al igual que uno usa al modelar un segment tree.\newline

Por lo tanto, podemos aplicar una operación de tipo consulta en el rango [a,b) y con eso obtenemos el valor máximo en el mismo y la posición en la que se encuentra.\newline

Cómo no solo buscamos el máximo, sino también el segundo más grande, esto equivale a actualizar el valor del máximo por el elemento neutro y volver a preguntarnos cual es el máximo en dicho rango. De esa manera tenemos los dos más grandes.\newline

\textbf{PERO OJO! Nos quedó desactualizada la estructura al haber agregado un valor neutro donde no iba. Hay que actualizar el valor del elemento en dicha posición con el resultado de la primer consulta.}\newline

Basta ahora sí, con responder la suma de los dos resultados obtenidos ante dichas operaciones ya que tenemos la estructura igual a como la teníamos antes de empezar esta pregunta.

\subsection{Respetando la complejidad}

El armado de un segment tree tiene como complejidad \bigo(D) siendo D la cantidad de elementos que vienen en la entrada.\newline

Esto ocurre pues la cantidad de posiciones que tendrá dicho arreglo (cantidad de nodos en el árbol) será 2 $\times$ (la potencia de 2 mayor o igual a D) Este número está en el orden D. \newline

Luego para cada pregunta en el intervalo se realizan dos llamadas a get de un segment tree y dos llamadas a set. Como bien vimos en clase, se realizan a lo sumo 4 preguntas por nivel del segment tree y al ser un árbol binario completo, tiene éste altura logarítmica y tenemos entonces a lo sumo 4 $\times$ log (D) operaciones de tipo constante (máximo de dos números, operaciones de mayor y retorno de funciones).\newline

Por lo tanto para cada pregunta, tardamos tiempo logarítmico, lo cual nos lleva a una complejidad de \bigo(R $\times$ log D)  \newline 

Como complejidad total, nos queda entonces la suma de dichas operaciones, lo que nos da \bigo(D + R $\times$ log D) como era pedido por enunciado.

\subsection{Testing}

Para probar la correctitud de dicho algoritmo, se decidió implementar un algortimo con complejidad \bigo(R $\times$ D) donde para cada intervalo, hacemos dos pasadas lineales buscando el máximo y el segundo más grande. \newline

Luego, se armaron casos de prueba con al menos 1000 elementos aleatorios en el arreglo y 1000 preguntas con rangos totalmente aleatorios que respetaran la condición de que contengan al menos dos elementos. El resultado de ambos algoritmos para todas las corridas fueron exitosamente iguales.

\subsection{Pseudocódigo}


\begin{algorithmic}
\Function{get}{segment_tree[], inicio, fin, N, limite_inferior, limite_superior}
	\If{inicio $\leq$ limite_inferior $\land$ fin $\geq$ limite_superior} \Comment{Si el intervalo está totalmente contenido}
		\State \Return segment_tree[N]
	\EndIf
	\If{fin $\leq$ limite_inferior $\lor$ inicio $\geq$ limite_superior} \Comment{Si el intervalo está afuera}
		\State \Return elemento_neutro
	\EndIf

	\Comment{Mitad adentro, mitad afuera}

	\State medio $\gets$ (limite_superior + minimo)/2;
	
	\State pair[int,int] parte_izquierda $\gets$ \textit{get(segment_tree, inicio, fin, 2*N, limite_inferior, medio)}
	\State pair[int,int] parte_derecha $\gets$ \textit{get(segment_tree, inicio, fin, 2*N+1, medio, limite_superior)}
	
	\If{parte_izquierda.first $>$ parte_derecha.first}
		\State \Return parte_izquierda
	\Else
		\State \Return parte_derecha
	\EndIf


\EndFunction
\end{algorithmic}
\hspace{.5cm}



\begin{algorithmic}
\Function{set}{segment_tree[], index, value,}
	\State arreglo[index].first $\gets$ value
	\State padre $\gets$ index/2
	\State mantuve $\gets$ false
	\While{padre$\geq$1 $\land \lnot$ mantuve}
		\If{arreglo[2*padre].first $>$ arreglo[2*padre+1].first}
		 	\State mantuve $\gets$ (arreglo[padre] == arreglo[2*padre]);
			\State arreglo[padre] $\gets$ arreglo[2*padre]; 	
		\Else
			\State mantuve $\gets$ (arreglo[padre] == arreglo[2*padre+1]);
		 	\State arreglo[padre] $\gets$ arreglo[2*padre+1];
		\EndIf
	\EndWhile 
\EndFunction
\end{algorithmic}
\hspace{.5cm}

\begin{algorithmic}
\Function{preguntar}{segment_tree[], limite_superior, inicio, fin}

	\State pair[int,int] mas_grande $\gets$ \textit{get(arreglo,inicio,fin, 1, 0, limite_superior/2)} \Comment{Busco el máximo}
	\State llevo $\gets$ mas_grande.first
	\State posicion_arreglo $\gets$ limite_superior / 2 + mas_grande.second
	\State \textit{set(arreglo,posicion_arreglo,valor_neutro)} \Comment{Pongo un elemento neutro en él.}
	\State pair[int,int] segundo_mas_grande $\gets$ \textit{get(arreglo,inicio,fin, 1, 0, limite_superior/2)} \Comment{Busco el segundo más grande.}
	\State \textit{set(arreglo,posicion_arreglo,mas_grande.first)} \Comment{Le restauro el valor.}
	\State llevo $\gets$ llevo + segundo_mas_grande.first \Comment{Tengo la suma de los dos más grandes.}
	\State \Return llevo

\EndFunction
\end{algorithmic}
\hspace{.5cm}

\begin{algorithmic}
\Function{armar_segment_tree}{segment_tree[], limite_superior}
	\Comment {Para cada nodo interior, le pregunto cual de ambos de los hijos son más grandes y me lo guardo.}
	\For{i $\gets$ (limite_superior/2)-1 .. 0 }
		\If{segment_tree[2*i].first $>$ segment_tree[2*i+1].first}
			\State segment_tree[i] $\gets$ segment_tree[2*i];
		\Else
			\State segment_tree[i] $\gets$ segment_tree[2*i+1];
		\EndIf
	\EndFor
\EndFunction
\end{algorithmic}
\vspace{.5cm}
\begin{algorithmic}
\Function{llenar_hoja}{segment_tree [], valor, posicion, limite_superior}
	\State segment_tree[limite_superior/2 + posicion] = {valor,posicion}
\EndFunction
\end{algorithmic}
\vspace{.5cm}
\begin{algorithmic}
\Function{main}{D, R, diversion[D], pregunta[R]}
	
	\State elemento_neutro $\gets$ [-1,-1]

	\State limite_superior $\gets$ 1


	\Comment{Busco la potencia de dos que sea más grande a D. Ese será la cantidad de hojas.}
	\While{limite_superior $<$ D}
		\State limite_superior $\gets$ limite_superior * 2
	\EndWhile

	\Comment{Tengo guardado entonces el tamaño del árbol binario completo}
	\State limite_superior $\gets$ limite_superior * 2 
	\State pair[int,int] segment_tree [limite_superior]

	\For{i $\gets$ 0 .. limite_superior} \Comment{Lleno el segment tree con todos valores neutros}
		\State segment_tree[i] $\gets$ elemento_neutro
	\EndFor
	
	\For{i $\gets$ 0 .. D} \Comment{Lleno las hojas con los valores correspondientes}
		\State \textit{llenar_hoja(segment_tree, diversion[i], i, limite_superior)}
	\EndFor

	\State \textit{armar_segment_tree(arreglo,limite_superior)} \Comment{Completo los nodos internos del árbol binario}

	\For{i $\gets$ 0 .. R}
		\State \textit{preguntar(arreglo, limite_superior, pregunta[i].inicio, pregunta[i].fin)}
	\EndFor
\EndFunction
\end{algorithmic}

\end{document}
