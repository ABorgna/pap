\section{Ejercicio 4}

\subsection{Enunciado}
El enunciado nos pide, dada una arreglo de tamaño N de matrices de 3x3 , un a matriz de 3x3 M y un numero L, decidir
si existe un subarreglo de largo L que al multiplicarlo nos de M.
Ademas todas las matrices del problema tienen sus elementos en Z100007

\subsection{Formalizando}
Reformulando, dado un arreglo a1...an de con elementos en Z3x3 100007, queremos ver si existen indices
i, j tal que j - i == L y ai x ai+1 x .... x a j-1 == M

\subsection{Algunas propiededades}
Lo primero que debemos recordar para resolver este problema es que, la multiplicacion de matrices es asociativa.
Como en muchos problemas en la computación, vamos a querer resolver ''subproblemas'' 
que nos ayuden a resolver el problema original. Así, nos será útil saber que si encontramos dos subarreglos tales que:
ai x ai+1 ... x ak = T1 y ak+1 x ... x aj = T2 y T1 * T2 = M, donde i y j 
respetan lo mencionado en el punto anterior..
encontramos lo que buscabamos.
\newline
Para ser más breves, denotemos como subarreglo \textbf{ganador} a un subarreglo que cumple las condiciones que buscamos.
Con la idea en mente de buscar subproblemas para resolver el problema original, dividamos el arreglo en dos.
Notamos que si al arreglo original lo dividimos, entonces se pueden dar los siguientes casos:
\begin{itemize}
\item Hay un subarreglo ganador totalmente contenido en la primer mitad.
\item Hay un subarreglo ganador totalmente contenido en la segunda mitad.
\item Hay un subarreglo ganador que esta parcialmente contenido en cada una de las mitades.
\item No hay subarreglo ganador, asi que la respuesta seria ''NO''.
\end{itemize}
Facilmente podemos ver que si resolvemos el tercer caso, es decir, cuando un subarreglo 
ganador esta parcialmente contenido en cada mitad, podriamos dar un algoritmo 
que utilice la técnica de ''Divide And Conquer'' para resolver el problema. 
