\section{Ejercicio 3}

El ejercicio consiste en encontrar la máxima suma de elementos contiguos de un array.
Esto es equivalente a encontrar un subarray que maximice la suma de sus elementos,
lo que se conoce como \textbf{Maximum Subarray Problem}.


Formalmente, dado un array $A = (a_0 \ldots a_{n-1})$ queremos encontrar el máximo de
$$
\sum_{s=i}^{j-1} a_s \;\;\;\; \text{con} \; 0 \leq i \leq j \leq n
$$


\subsection{Algoritmo y correctitud}

La idea es recorrer los prefijos $A_k = (a_0 \ldots a_{k-1})$ en orden de $k$ creciente
y para cada uno calcular el máximo de la suma dentro del array, $res_k$,
y la suma máxima entre los subarrays que terminen en el elemento $k$ o el subarray vacío, $border_k$.

Para el caso $k=0$, el array vacío, ambos valores son 0.

Para el caso $k+1$, podemos suponer que $res_k$ y $border_k$ tienen los máximos buscados.

$res_{k+1}$ puede usar o no el elemento $a_k$. Si no lo usa, $res_{k+1}$ es la suma máxima dentro del array $A_k$, esto es, $res_{k+1} = res_k$. Si lo usa, $res_{k+1}$ es $a_k$ sumado a la mejor suma posible entre los subarrays sufijos de $A_k$, que ya tenemos calculado como $border_k$. El valor de $res_{k+1}$ será por lo tanto el máximo entre $res_k$ y $a_k + border_k$.

Por otro lado, $border_{k+1}$ puede también usar o no $a_k$. Si no lo usa será $0$, la suma del array vacío. Si lo usa tomará como valor la suma de $a_k$ y la mejor suma posible entre los subarrays sufijos de $A_k$, que nuevamente es $border_k$. Finalmente, $border_{k+1}$ será el máximo entre $0$ y $a_k + border_k$.

Como $A = A_n$, nuestro resultado final será $res_n$.

\subsection{Pseudocódigo}

\begin{algorithmic}

\Function{sumaMaxima}{$array$}
    \State $res \gets 0$
    \State $border \gets 0$
    \For{$x$ in $array$}
        \State $border \gets max(0, border+x)$
        \State $res \gets max(res, border)$
    \EndFor
    \State \Return{$res$}
\EndFunction

\end{algorithmic}

\subsection{Complejidad}

El recorre una vez el array calculando los máximos descriptos en \bigo(1), resultando en una complejidad total de \bigo($n$).

