\section{Ejercicio 2}

\subsection{Enunciado}

\par{El enunciado nos dice que tenemos un grupo de A acciones, cuyos precios conocemos en D días. 
Vamos a agrupar las acciones en gráficos. Dos acciones pueden ir en un mismo gráfico, si las lineas de sus dibujos
como función no se tocan en ningún punto.}
\par{Nos piden que demos la mínima cantidad de gráficos necesaria para poder graficar todas las acciones, con 
la restricción previamente mencionada.} 

\subsection{¿Cuándo se cruzan dos acciones?}

\par{Vamos a notar que, el hecho de que dos acciones \textbf{NO} se crucen es equivalente a que 
haya una cuyos precios en los D días sean estrictamente mayores a los correspondientes de la otra.}

\par{Que dos acciones se crucen significa que, si tenemos un segmento A $\rightarrow$ B de una acción, 
y otro C $\rightarrow$ D de la otra y correspondientes al mismo día,  entonces A mayor a C y B menor a D, o viceversa.} 

\par{Si en los D días, una acción siempre domina a la otra (decimos que una ``está arriba'' de la otra ), entonces trivialmente no se cruzan.}

\par{Si no se cruzan, entonces no hay dos segmentos correspondientes al mismo día que se den vuelta. 
O A mayor a C y B mayor a D, o la inversa. Entonces una acción esta arriba de la otra. Afirmamos entonces que son equivalentes.}

\subsection{Definamos un orden}

\par{Nos interesa notar algunas propiedades de la relación \textbf{estar arriba}. Para empezar, 
la relación es transitiva. Si A está arriba de B, y B de C, A está arriba de C.
Si con un poco de cuidado, permitimos que dos elementos esten relacionados si sus precios son iguales en todo momento, entonces
la relación nos queda reflexiva y antisimétrica. Con lo cual, nos define un \textbf{orden parcial}. }

\par{Lo que el problema nos pide es una cantidad mínima de gráficos que necesitamos para dibujar todas las acciones. 
Por la equivalencia que vimos antes, gráficos de este estilo, son conjuntos de acciones donde no se cruzan, es decir, 
una esta arriba de otra, para dos cualesquiera del conjunto. Y como la relación es transitiva, existe una acción,
A_1 > A_2 > ... > A_k, en un conjunto del mismo gráfico, con k acciones.} 

\subsection{La gallinita dijo Eureka}

\par{Por la observación, vemos que lo que estamos buscando es la mínima cantidad de \textbf{cadenas} que necesitamos
para cubrir un conjunto parcialmente ordenado. Donde una cadena, es un conjunto totalmente ordenado.}

\par{Sabemos además por el \textbf{Teorema de Dilworth} que esta mínima cantidad de cadenas es igual a la mínima anticadena
(un conjunto de elementos incomparables), y que estas se pueden calcular mediante un Flujo Máximo!}

\par{Usamos esta construcción de grafo de Dilworth, para calcular la mayor anticadena. El grafo que nos queda 
tiene para cada acción a $in$ Acciones, dos nodos a$'$, a$''$ $in$ V. Por otro lado, tenemos dos nodos, s y t (source y sink del nuevo grafo).
s esta conectado con todos los a$'$, con aristas de capacidad 1. Y t esta conectado con todos los a$''$ con aristas de capacidad 1.
Por último, todos los si tenemos dos acciones a,b y a \textbf{está arriba} de b, entonces a$'$ esta conectada con b$''$ con una arista de capacidad 1.
Si f es un flujo máximo de s a t en este nuevo grafo, por el teorema de Dilworth, la mayor anticadena es igual a $|acciones| - |f|$ = $A - |f|$, y es lo que buscábamos.  
}

\subsection{Complejidad}

\par{La primer parte de nuestra solución, para construir el grafo, necesita ver si dadas dos acciones, 
una esta arriba de la otra. Dadas dos acciones, chequear esto nos tarda \bigo(D), simplemente iterando por los días.
Por eso esta primer parte es \bigo($A^2 \times D $)}

\par{Luego, tenemos que buscar el flujo máximo. Utilizamos el algoritmo de Edmonds-Karp, que en cada iteración hace un BFS.
Pero utilizamos la cota que nos dice que la cantidad de iteraciones esta acotada por el flujo máximo. 
El flujo máximo es A, por construcción del grafo (las aristas que salen de s).
Entonces sabemos que la complejidad del algoritmo es \bigo($ (E + V) \times A$), donde E y V son las aristas y nodos respectivamente.
Ademas estas son \bigo($A^2$) y \bigo($A$) respectivamente, por lo cual el algoritmo de flujo termina siendo \bigo($ (A^2 + A) \times A$) 
= \bigo($ (A^2) \times A$). }

\par{Entonces, en total todo el algoritmo es \bigo($ A^2 \times (A + D) $) que es exactamente lo pedido.}
