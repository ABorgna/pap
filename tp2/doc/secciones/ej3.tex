\section{Ejercicio 3}

En este problema tenemos una ciudad representada por esquinas y calles bidireccionales entre ellas, de forma tal que se puede ir desde cualquier esquina a cualquier otra. A continuación nos realizan varias preguntas que pueden ser de tres tipos:

\begin{itemize}

    \item \textbf{A}: contar la cantidad de calles que al ser cortadas desconectan dos esquinas dadas
    \item \textbf{B}: decidir si al cortar una calle dada quedan desconectadas dos esquinas cualquiera
    \item \textbf{C}: dada una esquina, contar la cantidad de esquinas que siempre seguirían conectadas a ella aunque se corte una calle cualquiera

\end{itemize}

\subsection{Modelado del problema}

La ciudad se puede modelar fácilmente con un grafo $G$ donde cada esquina es un vértice y hay una arista por cada calle, conectando los nodos correspondientes a sus esquinas. Como siempre hay un camino entre dos esquinas, este grafo es conexo.

A partir de esta representación se puede traducir los tres tipos de preguntas:
\\

\noindent Tipo \textbf{A}:

Dados los dos vértices $u$ y $v$, consideramos todos los caminos simples entre ellos.
Si existe una arista que pertenezca a todos los caminos, al cortar esta no habrá un camino entre $u$ y $v$.
Si en cambio se corta una arista que no pertenecía a algún camino, este camino seguirá existiendo y por lo
tanto $u$ y $v$ seguirán conectados.

Mas aún, si una arista $e$ perteneciente a un camino $P$ entre $u$ y $v$ estuviera contenida en un ciclo del grafo, significa que existía un camino $P_e$ entre los vértices incidentes a $e$ que no pasa por este. Si tomamos el camino resultante de reemplazar $e$ por $P_e$ en $P$ (y obtener el camino simple contenido de ser necesario), obtendremos un camino simple entre $u$ y $v$ que no pasa por $e$. Si en cambio $e$ está en un camino simple entre $u$ y $v$ pero no pertenecía a algún ciclo de $G$, todo camino entre $u$ y $v$ tendrá que pasar por él ya que sino podríamos construirnos un ciclo que lo contenga.

Por lo tanto, cortar $e$ desconectará $u$ y $v$ si y solo si $e$ estaba en un camino simple entre estos y $e$ no pertenecía a un ciclo de $G$ (esto es, $e$ es un puente). Finalmente, la respuesta a la pregunta será la cantidad de puentes en un camino simple desde $u$ hasta $v$.
\\

\noindent Tipo \textbf{B}:

Dada una arista $e$, quiero saber si al cortar $e$ se desconectan dos vértices cualesquiera del grafo. Esto es equivalente a preguntar si $e$ es un puente.
\\

\noindent Tipo \textbf{C}:

Dado un vértice $u$, quiero contar la cantidad de nodos a los que siempre llego cuando saque una arista cualquiera del grafo. Esto es, quiero contar la cantidad de nodos en la componente 2-edge conexa de $u$ (menos el propio $u$).

\subsection{Algoritmo}

La idea para resolver el problema es precalcular a partir del grafo los puentes y el tamaño de la componente 2-edge conexa a la que pertenece cada vértice.
\\

Para ello usamos el algoritmo para buscar componentes biconexas y puentes visto en clase, modificándolo para que calcule también componentes 2-edge conexas. Esto se logra gracias a que las componentes 2EC del grafo están separadas siempre por aristas puente (se puede ver trivialmente a partir de la definición que una arista es puente sii los nodos adyacentes pertenecen a componentes 2EC diferentes).

Para ello mantenemos una pila con los nodos visitados que no fueron agregados a una componente 2EC, agregando los nodos que vamos visitando. Cuando detectamos un puente, sabemos que todos los nodos sin marcar a partir del que estamos explorando pertenecen a una nueva componente y por lo tanto los desencolamos y guardamos el tamaño de la componente a la que pertenecen en un array. Ademas debemos realizar esto con los nodos que queden sin extraer al finalizar el dfs.
\\

Teniendo estos valores, podemos responder una pregunta de tipo \textbf{A} buscando un camino simple cualquiera desde $u$ a $v$ usando DFS mientras contamos la cantidad de puentes por los que pasamos.
Las preguntas de tipo \textbf{B} serán directamente ver si la arista dada era un puente.
Y por último, las preguntas \textbf{C} se responden leyendo el tamaño de la componente 2-edge conexa del nodo y restándole 1 (para no contar el propio nodo).

\subsection{Pseudocódigo}

\algdef{SE}[DOWHILE]{Do}{doWhile}{\algorithmicdo}[1]{\algorithmicwhile\ #1}%

\begin{algorithmic}

\Function{precalcular}{}
    \State inicializar el array $esPuente$ todos en false
    \State inicializar el array $tamanoComponente2EC$ todos en 0

    \State $verticesSinComponente : stack de vertices$
    \State $depth : array de enteros$
    \State $low : array de enteros$

    \State $dfsPrecalculo(0, 0, -1, verticesSinComponente, depth, low)$
\EndFunction
\\

\Function{dfsPrecalculo}{$v : int, d : int, padre : int, verticesSinComponente : stack<int>,$
                         $depth : array<int>, low : array<int>$}
    \State $depth[v] \gets d$
    \State $low[v] \gets d$
    \State encolar $v$ en $verticesSinComponente$

    \For{$u$ vecino de $v$ diferente a $padre$ con arista $e$}
        \If{$depth[u] = -1$}
            \State \textbackslash\textbackslash Es un tree edge
            \State $posicionEnElStack \gets verticesSinComponente.size()$
            \State $dfsPrecalculo(u, d+1, v, verticesSinComponente, depth, low)$
            \State $low[v] \gets min(low[v], low[w])$

            \If{$low[u] \geq depth[u]$}
                \State marcar $e$ como $puente$
                \State \textbackslash\textbackslash
                    Guardamos la componente 2EC que se cierra con el puente
                \State $tamano \gets
                    verticesSinComponente.size() - posicionEnElStack$
                \Do
                    \State $top \gets $tope de $verticesSinComponente$
                    \State $tamanoComponente2EC[top] \gets tamano$
                    \State desapilar el tope de $verticesSinComponente$
                \doWhile{$top$ es diferente a $u$}
            \EndIf
        \Else
            \State \textbackslash\textbackslash Es un back edge
            \State $low[v] \gets min(low[v], depth[u])$
        \EndIf
    \EndFor
    \If{no tenía $padre$}
        \State $tamano \gets verticesSinComponente.size()$
        \Do
            \State $top \gets $tope de $verticesSinComponente$
            \State $tamanoComponente2EC[top] \gets tamano$
            \State desapilar el tope de $verticesSinComponente$
        \doWhile{$verticesSinComponente$ tenga elementos}
    \Else
\EndFunction
\\

\end{algorithmic}

\begin{algorithmic}

\Function{queryA}{$u : int, v : int$}
    \State inicializar el array $visitados$ todos en false
    \State \Return $dfsPuentes(u,v).puentes$
\EndFunction
\\

\Function{dfsPuentes}{$u : int, destino : int$}
    \State Marcar $u$ como $visitado$
    \If{$u$ es el $destino$}
        \State \Return $\{porAca: true, puentes: 0\}$
    \EndIf
    \For{$v$ vecino no $visitado$ de $u$}
        \State $res \gets dfsPuentes(v, destino)$
        \If{$res.porAca$}
            \If{$(uv)$ es puente}
                \State \Return $\{porAca: true, puentes: res.puentes + 1\}$
            \Else
                \State \Return $\{porAca: true, puentes: res.puentes\}$
            \EndIf
        \EndIf
    \EndFor
    \State \Return $\{porAca: false, puentes: 0\}$
\EndFunction
\\

\end{algorithmic}

\begin{algorithmic}

\Function{queryB}{$e : int$}
    \State \Return{$e$ es puente}
\EndFunction
\\

\end{algorithmic}

\begin{algorithmic}

\Function{queryC}{$u : int$}
    \State \Return{tamano componente 2EC de $u - 1$}
\EndFunction

\end{algorithmic}

\subsection{Complejidad}

TODO: precómputo
\\

Luego del precómputo, saber si una arista es puente o el tamaño de la componente 2EC de un nodo es simplemente acceder a un array y por lo tanto se pueden hacer en \bigo(1).

Por cada query \textbf{A}, realizamos un DFS con complejidad \bigo($M$). Y para las queries \textbf{B} y \textbf{C} hacemos solamente un acceso a un array y por lo tanto terminamos en tiempo \bigo(1).

Finalmente, si $Q_A$, $Q_B$ y $Q_C$ representan las cantidades de queries de cada tipo, la complejidad final será el resultado de realizar el precómputo y luego las queries necesarias, \bigo($M + M Q_A + Q_B + Q_C$).
