\section{Ejercicio 2}

En este problema queremos calcular la cantidad esperada de permutaciones que realiza el siguiente algoritmo
hasta ordenar un array dado.

\begin{enumerate}
    \item Comenzamos con $i = 0$ y el arreglo $A$ de tamaño $N$ dado en la entrada.

    \item Permutamos al azar el arreglo $A[i..N)$.

    \item Mientras que $i$ sea menor a $N$ y el menor elemento de $A[i..N)$
            sea $A[i]$ incrementamos $i$ en $1$.

    \item Si $i < N$ volvemos al paso $2$.
\end{enumerate}

Como siempre se realiza una permutación aleatoria de todo el arreglo al iniciar,
no nos importa el orden inicial de $A$.
Llamaremos $B$ al array que queremos obtener, con los elementos de $A$ ordenados.

\subsection{La solución}

El algoritmo presentado va incrementando $i$ de a uno siempre manteniendo el invariante
de que $A[0..i)$ ya está ordenado, o sea que $A[0..i) = B[0..i)$ y por lo tanto
los elementos de $A[i..N)$ serán los mismos que $B[i..N)$.

Llamaremos $X$ a la cantidad total de permutaciones que se realizan y
$X_i$ a la cantidad de permutaciones realizadas durante el paso $i$.
$X = 1 + \sum_{i=0}^{N-1} X_i$ (ya que siempre se realiza una permutación al inicio).
\\

Luego de cada permutación, la probabilidad de que haya quedado en la posición $i$ un elemento
menor o igual a todos los siguientes la podemos calcular como

$$
p_i =
\frac{\#\ de\ elementos\ minimos\ en\ A[i..N)}{\#\ de\ elementos\ en\ A[i..N)} =
\frac{\#\{x \in B[i..N) \;|\; \forall x' \in B[i..N), \; x \leq x' \}}{N - i}
$$

Como siempre al empezar el paso $i$ los elementos de $A[i..N)$ ya se encuentran distribuidos uniformemente,
queremos contar la cantidad de permutaciones \textit{fallidas} (que no colocaron un elemento mínimo al principio). Por lo tanto $X_i$ tendrá una distribución geométrica que contará la cantidad de fallos,
donde cada ensayo tiene probabilidad $p_i$.

Por propiedad de la distribución geométrica, $E(X_i) = \frac{1}{p} - 1$.
\\

Queremos calcular la esperanza del total de permutaciones del array, $E(X) = E(1 + \sum_{i=0}^{N-1} X_i)$.
Por la linealidad de la esperanza, $E(X) = 1 + \sum_{i=0}^{N-1} E(X_i) = 1 + \sum_{i=0}^{N-1} (\frac{1}{p_i} - 1)$.

Sabiendo esto, para calcular el resultado solo necesitamos ir recorriendo el array ordenado $B$
acumulando en cada paso el $E(X_i)$ correspondiente.

\subsection{El algoritmo}

Como para calcular $p_i$ necesitamos conocer la cantidad de veces que se repite el menor elemento,
ordenamos inicialmente $A$.

Luego recorremos cada posición $i$ del array ordenado,
manteniendo siempre un valor $j \geq i$ que indica la posición del primer elemento diferente a $A[i]$.

$j$ se inicializa en 0 y al comenzar cada ciclo lo actualizamos realizando una búsqueda lineal
en $A$ a partir de $j$, deteniéndonos cuando encontramos un elemento diferente a $A[i]$.
Si en algún momento $j = i$ al inicializar el ciclo necesariamente tendremos que incrementarlo ya que $A[i] = A[i]$, y como $i$ se incrementa de a $1$ por ciclo nunca puede quedar $j<i$.

Solo resta calcular $p_i$ y sumárselo al resultado. La cantidad total de elementos restantes la calculamos como $N - i$. Y la cantidad de elementos mínimos, como el array está ordenado, será $j - i$.
\\

A continuación se muestra el pseudocódigo del algoritmo:

\newpage

\begin{algorithmic}

\Function{calcularEsperanza}{$a : vector<int>$}{$\; \rightarrow float$}
    \State ordenar $a$
    \State $res \gets 1$
    \State $j \gets 0$
    \For{$i$ entre $0$ y $n-1$}
        \State $j \gets$ buscar el indice del primer elemento $\neq a[i]$ a partir de $j$
        \State $res \gets res + (j-i)/(n-i)$
    \EndFor
    \State \Return $res$
\EndFunction

\end{algorithmic}

\subsection{Complejidad}

Primero debemos leer la entrada en tiempo \bigo($n$).

Luego ordenamos $a$ en \bigo($n\ log(n)$).

El ciclo se ejecuta $n$ veces y el nuevo resultado se calcula en \bigo(1),
por lo que nos quedaría ver la complejidad de actualizar $j$.

Cada búsqueda local compara una vez contra el elemento $j$ y luego, si este era igual al $i$-ésimo,
realiza $k_i \geq 0$ comparaciones mas hasta que finalmente encuentra un elemento diferente y aumenta
$j$ en $k_i$. Como $j$ está acotado por $n$, la suma de todas las $k_i$ no supera este valor y por lo
tanto entre todas las búsquedas se realizan \bigo($n$) operaciones.

Finalmente, sumando todas las complejidades, el algoritmo corre en \bigo($n\ log(n)$).
